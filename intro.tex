\chapter{Introduction}
\section{Background of the Study}
Zinc oxide is a metal oxide semiconductor that has many different uses based on its structure such as in thin film electronics.
Zinc oxide thin films are used in a wide variety of applications as gas sensors \cite{florido17} and as photovoltaic cells \cite{fetsense}.
In order for the material to perform predictably, it is necessary to be able to correlate processing parameters to structural characteristics and property manifestations \cite{florido17, fetsense}.

There are different methods for thin film fabrication including successive ionic layer adhesion and reaction processes \cite{florido17}.
These processes are a family of processes consisting of successive ionic material depositions onto a substrate and thermal annealing \cite{gao08}.
These process have been considered to be studied by researchers due to the processes being cost-effective and easily modifiable \cite{gao08, vargas08, vargas2}.
There are three studied fabrication paramaters of interest which include the ionic precursor solutions, number of deposition cycles and annealing temperature \cite{gao08, vargas08}.
By varying these parameters, it has been shown, by the aforecited studies that there are differences in thin film structure and property.
Aside from empirical conclusions as of the studies, there seems to be little research done on how processing affects material structure and property.

Due to the promise of rapid, high-throughput materials processing brought by data science driven approaches to materials processing through materials knowledge systems (MKS) frameworks \cite{gupta15, sun17, yabansu14} the field of materials informatics has grown.
The MKS framework, as defined by \citeA{yabansu14}, is dependent on the definition of a microstructure function $m_r^n$, representing the probability density of finding a specific local state $n \in N$, characterized by physical and chemical properties, at a spatial bin $r \in R$.
A number of studies have proposed the MKS to use discretized microstructure functions in order to apply statistical measures \cite{gupta15, sun17, yabansu14}.
Numerous studies regarding the use of MKS frameworks have produced different microstructure functions depending on the application of the materials of interest ranging from two to ten and more local states \cite{gupta15, sun17, yabansu14}. 

Currently, there exists an implementation of the MKS framework in the Python language (PyMKS) \cite{pymks}.
The said framework implementation is maintained by researchers Georgia Institute of Technology and has been developed by a number of contributors to the repository.
It is currently available on Github as an open-source project licensed under the MIT license.
The Julia languate is another computational programming language that currently gains popularity on Github \cite{julia15}.

\section{Objectives of the Study}
The study aims to compare the structure of zinc oxide thin films fabricated using three different successive ionic layer adhesion and reaction processes.
Specifically, the study will attempt to cover and discuss necessary measures that will aid in the creation of effective discrete spatial statistics of the particles, particle boundaries, and null spaces within the material structure, which will be used to find correlations to manipulations in the number of deposition cycles and annealing temperature.

The study also aims to be able to create a MKS methodology template in the Julia programming language.
Specifically, the study aims to provide a new platform for materials informatics using the features of the programming language.

\section{Significance of the Study}
Some of the numerous applications of zinc oxide thin films are dependent on the different distribution-dependent properties of the material, including thin film sheet resistance, porosity, and transmittance.
The success of the study will indicate that there is a promise for growth in thin film semiconductor processing and analysis in diverse applications by reducing the amount of human effort, resources, and time needed.
The expression of the MKS framework, case-dependent or independent, in multiple languages allows for the growth of the framework into several applications allowing open source communities to further develop the capabilities of the framework.

\section{Scope and Limitations}
The study focuses on the comparison of thin films fabricated using three different SILAR processes varying in two processing parameters: number of deposition cycles and annealing temperature.
The thin films were only characterized using optical microscopy, x-ray diffraction, and four-point probe sensing.
Despite the number of characterization methods used, the structures of the differentially fabricated thin films were not fully elucidated.
Since there is no implementation of the MKS framework in the programming language of interest, the analyses and code were based on the biasses of the researcher on the field.
