% abstract of the report

\begin{abstract}
    Zinc oxide has become the metal oxide semiconductor of interest in similar studies due to its electrical properties.
    Materials knowledge frameworks have been shown to create process-structure-property linkages in the development of different material structures.
    The study aims to compare the the structure of differentially fabricated thin films and the find a correlaion between particle distribution and film resistivity.
    Three different SILAR treatments with varying number of deposition cycles and annealing temperatures were applied to untreated glass substrates.
    The fabricated thin films were characterized through optical microscopy, x-ray diffraction, and four-point probe sensing.
    Materials fabricated for two processes have been shown to be zinc oxide from respective diffraction results.
    There is too few material adhered onto a film fabricated using the SILAR process with less deposition cycles; thus, material phase was not verified.
    Discrete microstructure functions were derived from particle distribution (presence/absence) within the films.
    Two-point spatial correlations were used in principal component analysis to determine structural differences.
    Image analysis results indicate that particles clump in an area, and that the distribution of the clumps are sparse and random.
    There exists a correlation between material structure and property ($R^2 = 0.987$), but it may be an overfitting model since distributions were sparse and random.
    It is suggested to study material distribution and property in films using glass treated in sulfuric acid as substrate.
\end{abstract}
