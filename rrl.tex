\chapter{Review of Related Literature}

\section{Zinc oxide}

\subsection{Crystal structure}

One of the key characteristics governing most of the properties of condensed matter is the crystal structure of a material.
The crystal structure is thus important in characterizing material.

Zinc oxide is a metal oxide semiconductor having a hexagonal crystal structure.
In studies such as those of \citeA{gao08} and \citeA{fdoping}, there are three identified main peaks in the diffaction pattern (see Characterization Methods) of zinc oxide.
These three peaks are the first three peaks in the diffraction pattern of ZnO found roughly at $2\theta \in \{32\degg , 34\degg , 36\degg\}$.
The planes of diffraction are $(100)$, $(002)$, and $(101)$, respectively, with the highest peaks at the $(002)$ plane suggesting preferrential orientation in the \textit{c}-axis.
With regard to flat surfaces, this suggests formation of vertical rod-like structures from the substrates.

Numerous studies such as those of \citeA{gao08}, \citeA{vargas08}, and \citeA{fdoping} have shown experimentally that this structure of zinc oxide can be achieved even through production using different processes.
All of these showed that manipulating the processing parameters have various effects on the crystalline structure of the material.
They have effects on the crystallite size and relative intensities (diffraction).
It is worth noting that the processes used for fabrication are SILAR processes, as will be described in a later section.

The three SILAR processes analyzed by \citeA{vargas08} showed differences in the relative intensities of the diffraction peaks.
It can also be noted that some of the diffraction peaks present in films fabricated by a process were not found in another.
\citeA{fdoping} showed how different doping concentrations affected the overall crystallinity of the thin film structure.
The results, as expected, showed that different processing parameters lead to different material crystal structures.

\subsection{Electrical properties}

Studies have shown that zinc oxide thin films exhibit a wide variety of interesting properties such as  \emph{band gap} and \emph{electrical resistivity}.
Those of \citeA{fdoping} and \citeA{vargas08} made empirical studies on the effects of different processes on the electrical properties of ZnO.
Specifically, \citeA{fdoping} compared band gaps and electrical resistivity, while \citeA{vargas08} compared band gap.
Just as different processing parameters have led to differences in material crystal structures, the difference in crys

Research on zinc oxide thin films as gas sensors by exploiting the reversible reactions of reducing gases to the surface of the material \cite{florido17}.
These surface reactions change the movement of charges within the crystal structures of materials affecting their \emph{resistivity}.
There is more literature that shows how crystal structures affect the electrical resistivity and gas sensing response of metal oxide semiconductors from theoretical studies such as of \citeA{dey} and \citeA{powerlaw} to \textit{in silica} studies like involving DFT \cite{dft}; however, these will not be focused on in this study.

The importance of being able to computationally predict the properties similar to the \textit{in silica} study referred to above is a huge milestone in predicting properties given the structure of a material.

\subsection{Successive ionic layer adhesion and reaction processes}

Successive ionic layer adhesion and reaction (SILAR) processes are a family of processes wherein materials are deposited onto substrates by means of successive surface reactions.
The processes described will be based on the study by \citeA{gao08}.

The processes consist of two main phases: a \emph{deposition} phase and an \emph{annealing} phase.
The deposition phase in material fabrication involves a set of precursor solutions: cationic solutions, complexing solutions, and anionic solutions, to perform the reactions on the surface of the substrate.
The three solutions are aqueous \cite{gao08}.
Thus, the substrate surface must be hydrophilic for the solutions to react in the and adhere.
However, naturally, glass is hydrophobic which can be noted by the formation of droplets on its surface.
\citeA{gao08} suggested the following methodology to develop the hydrophilic property of glass surface: boil the substrates in dilute $H_2SO_4$ ($1:10$ v/v) for $30$ minutes, then completely rinse the substrates in ethanol, acetone, and de-ionized water.
Similar processes were also done by \citeA{fdoping} and \citeA{vargas08}.

A study conducted by \citeA{vargas2} was performed in order to determine the effects of three different number of deposition cycles to thin film crystallinity.
It was observed that the relative intensities of the diffraction peaks drastically increased with the number of deposition cycles.
However, there was no literature, as of this manuscript, that can be used to explain this phenomenon.
It should be noted that the studies conducted by \citeA{vargas2} and \citeA{vargas08} differ in the parameter tested in SILAR processing.

In addition to the precursor solutions and substrate properties described, another factor that affects the output of SILAR processes is the drying interval being deployed between each dipping processes \cite{gao08}.
In the cited literature, the researchers compared the effects of two drying processes in the crystallinity of the thin films.
In applying a drying process of $3-5$ or $30$ seconds between each deposition \emph{cycle}, the resulting thin films have a defined and \emph{periodic} microstructure.
Those films processed with no drying process between each deposition cycle were found to have \emph{amorphous} microstructures.

The annealing process in SILAR processes is crucial for the growth of crystal structures in the thin film.
According to \citeA{fdoping}, it is still unknown how, specifically, annealing affects this growth.
There are numerous \emph{empirical} studies regarding this topic, however, there is currently no solid theory known to the researchers as of date.

In the study of \citeA{fdoping}, it can be observed through the XRD spectra that annealing plays a huge role in the development of crystal structures.
It can be seen that the measured intensities at the Bragg angles stated above have increased.
This means that better crystallinity is achieved through thermal annealing.
The study of \citeA{gao08} has found that there can also be different effects on thin film crystallinity based on the annealing environment used.
It can be inferred from the data that the adjusting the annealing environment leads to different average crystallite sizes within the thin films.

Studies by \citeA{vargas08}, \citeA{vargas2}, \citeA{gao08}, and \citeA{fdoping} all have used different precursor solutions in preparing thin films.

\subsection{Characterization methods}
\subsubsection{X-ray diffraction}
\subsubsection{Optical microscopy}
\subsubsection{Scanning electron microscopy}
\subsubsection{Four point probe sensing}

\section{Multiscale Modeling}

\subsection{Representative volume elements}
\subsection{Finite element methods}
\subsection{Deep learning methods}
\subsection{Property localization}
\subsection{Property homogenization}

\section{Materials knowledge systems}

\subsection{Microstructure functions}
\subsection{One-point spatial correlations}
\subsection{Two-point spatial correlations}
\subsubsection{Calculation of values}
\subsubsection{Low dimensional space representation}
\paragraph{Principal component analysis}
\paragraph{Partial least squares analysis}

\section{Julia}

\subsection{Comparison with currently used languages}
\subsection{Language features}
\subsubsection{Data representation}
\subsubsection{Data types}
\subsubsection{Multiple dispatch}
\subsubsection{Parallel programming}
\subsection{Syntax}
\subsubsection{Function representation}
\subsubsection{Mathematical representation}
\subsection{Published computational studies and applications}
\subsubsection{NESSie.jl}
\subsubsection{JuMP.jl}
\subsubsection{BioSimulator.jl}
\subsubsection{QuantumOptics.jl}
