\chapter{Review of Related Literature}

\section{Zinc oxide}

\subsection{Crystal structure}

One of the key characteristics governing most of the properties of condensed matter is the crystal structure of a material.
The crystal structure is thus important in characterizing material.

Zinc oxide is a metal oxide semiconductor having a hexagonal crystal structure.
In studies such as those of \citeA{gao08} and \citeA{fdoping}, there are three identified main peaks in the diffaction pattern (see Characterization Methods) of zinc oxide.
These three peaks are the first three peaks in the diffraction pattern of ZnO found roughly at $2\theta \in \{32\degg , 34\degg , 36\degg\}$.
The planes of diffraction are $(100)$, $(002)$, and $(101)$, respectively, with the highest peaks at the $(002)$ plane suggesting preferrential orientation in the \textit{c}-axis.
With regard to flat surfaces, this suggests formation of vertical rod-like structures from the substrates.

Numerous studies such as those of \citeA{gao08}, \citeA{vargas08}, and \citeA{fdoping} have shown experimentally that this structure of zinc oxide can be achieved even through production using different processes.
All of these showed that manipulating the processing parameters have various effects on the crystalline structure of the material.
They have effects on the crystallite size and relative intensities (diffraction).
It is worth noting that the processes used for fabrication are SILAR processes, as will be described in a later section.

The three SILAR processes analyzed by \citeA{vargas08} showed differences in the relative intensities of the diffraction peaks.
It can also be noted that some of the diffraction peaks present in films fabricated by a process were not found in another.
\citeA{fdoping} showed how different doping concentrations affected the overall crystallinity of the thin film structure.
The results, as expected, showed that different processing parameters lead to different material crystal structures.

\subsection{Electrical properties}

Studies have shown that zinc oxide thin films exhibit a wide variety of interesting properties such as  \emph{band gap} and \emph{electrical resistivity}.
Those of \citeA{fdoping} and \citeA{vargas08} made empirical studies on the effects of different processes on the electrical properties of ZnO.
Specifically, \citeA{fdoping} compared band gaps and electrical resistivity, while \citeA{vargas08} compared band gap.
Just as different processing parameters have led to differences in material crystal structures, the difference in crystal structures also reflect differences in material property.

Research on zinc oxide thin films as gas sensors by exploiting the reversible reactions of reducing gases to the surface of the material \cite{florido17}.
These surface reactions change the movement of charges within the crystal structures of materials affecting their \emph{resistivity}.
There is more literature that shows how crystal structures affect the electrical resistivity and gas sensing response of metal oxide semiconductors from theoretical studies such as of \citeA{dey} and \citeA{powerlaw} to \textit{in silica} studies like involving DFT \cite{dft}; however, these will not be focused on in this study.

The importance of being able to computationally predict the properties similar to the \textit{in silica} study referred to above is a huge milestone in predicting properties given the structure of a material.

\subsection{Successive ionic layer adhesion and reaction processes}

Successive ionic layer adhesion and reaction (SILAR) processes are a family of processes wherein materials are deposited onto substrates by means of successive surface reactions.
The processes described will be based on the study by \citeA{gao08}.

The processes consist of two main phases: a \emph{deposition} phase and an \emph{annealing} phase.
The deposition phase in material fabrication involves a set of precursor solutions: cationic solutions, complexing solutions, and anionic solutions, to perform the reactions on the surface of the substrate.
The three solutions are aqueous \cite{gao08}.
Thus, the substrate surface must be hydrophilic for the solutions to react in the and adhere.
However, naturally, glass is hydrophobic which can be noted by the formation of droplets on its surface.
\citeA{gao08} suggested the following methodology to develop the hydrophilic property of glass surface: boil the substrates in dilute $H_2SO_4$ ($1:10$ v/v) for $30$ minutes, then completely rinse the substrates in ethanol, acetone, and de-ionized water.
Similar processes were also done by \citeA{fdoping} and \citeA{vargas08}.

A study conducted by \citeA{vargas2} was performed in order to determine the effects of three different number of deposition cycles to thin film crystallinity.
It was observed that the relative intensities of the diffraction peaks drastically increased with the number of deposition cycles.
However, there was no literature, as of this manuscript, that can be used to explain this phenomenon.
It should be noted that the studies conducted by \citeA{vargas2} and \citeA{vargas08} differ in the parameter tested in SILAR processing.

In addition to the precursor solutions and substrate properties described, another factor that affects the output of SILAR processes is the drying interval being deployed between each dipping processes \cite{gao08}.
In the cited literature, the researchers compared the effects of two drying processes in the crystallinity of the thin films.
In applying a drying process of $3-5$ or $30$ seconds between each deposition \emph{cycle}, the resulting thin films have a defined and \emph{periodic} microstructure.
Those films processed with no drying process between each deposition cycle were found to have \emph{amorphous} microstructures.

The annealing process in SILAR processes is crucial for the growth of crystal structures in the thin film.
According to \citeA{fdoping}, it is still unknown how, specifically, annealing affects this growth.
There are numerous \emph{empirical} studies regarding this topic, however, there is currently no solid theory known to the researchers as of date.

In the study of \citeA{fdoping}, it can be observed through the XRD spectra that annealing plays a huge role in the development of crystal structures.
It can be seen that the measured intensities at the Bragg angles stated above have increased.
This means that better crystallinity is achieved through thermal annealing.
The study of \citeA{gao08} has found that there can also be different effects on thin film crystallinity based on the annealing environment used.
It can be inferred from the data that the adjusting the annealing environment leads to different average crystallite sizes within the thin films.

Studies by \citeA{vargas08}, \citeA{vargas2}, \citeA{gao08}, and \citeA{fdoping} all have used different precursor solutions in preparing thin films.

\section{Multiscale Modeling}

\subsection{Representative volume elements}

Crucial in most multiscale modeling methodologies is the development of elements of finite volume that reflects the general physics of the macrostructure.
Representative volume elements (RVE) are images of the material chosen such that the overall distribution of the \emph{local states} and their interactions may be able to represent the underlying mechanisms (physics) of the whole structure at a particular length scale.
From here onward, the terms regarding RVEs will be defined as follows:

\begin{enumerate}
  \item Probe volume ($S$) - \emph{material volume} being studied (same as RVE)
  \item Voxel ($s$) - \emph{small, discrete} volume elements taken from an RVE ($s \in S$)
  \item Local state space ($H$) - set of all realizable physical and chemical properties representing material characteristics at a \emph{particular length scale}
  \item Local state ($h$) - an element of the local state space ($h \in H$)
  \item Boundary conditions ($\aleph$) - a set of requirements for interactions within $S$
\end{enumerate}

These definitions are important in establishing concepts regarding the use of RVEs in different property homogenization models.
As will be seen in later discussions, $S$ and $\aleph$, but not $s, H, h$, will have slightly varied definitions depending on the data needed by specific modeling techniques.

\subsection{Finite element methods}

Finite element methods rely on computer simulations of mechanical theories on finite, discrete elements $s \in S$.
This means that this method can be used to simulate and study the effects of mechanical systems with respect to space and their interactions with one another.
An interesting study done by \citeA{bone} dealt with finding an optimal $\aleph$, with regard to the case, conditions to be met incorporating the interaction of collagen fibers ($h_1 \in H$) and the surrounding cellular matrix ($h_2 \in H$).
In finite element considerations, however, these are not the only conditions that must be followed.
Energy conformation conditions also have to be observed \cite{cosserat}.

The work of \citeA{cosserat} dealt with the establishment of a mathematical derivation of a model for microstate property homogenization.
Their work, split into six sections of rigorous mathematical manipulation, involves these steps: expression of Cauchy continuum energy and mechanical conformation conditions into Cosserat continuum conditions, the development of a homogenization model, and establishment of $\aleph$.
The steps in proof done by these researchers present a method in developing homogenization models using different representations of an RVE.
It is worth noting that the definition of the RVE is flexible for as long as proper conditions are imposed.

These finite element methods are \emph{very} effective in modeling the interactions of different materials and their configurations within an RVE.
Due to the complexity of the implementation of this method, computer simulations may run for hours, which is not efficient in rapid, high-throughput materials processing.

\subsection{Deep learning methods}

With the development of the various fields of artificial intelligence, many different modeling techniques have been developed, especially in the areas of image recognition, and multivariable property prediction with the use of convolutional and artifical neural networks.
A homogenization model was developed by \citeA{cnn} for predicting the ionic conductivity of ceramics using convolutional neural networks.
The research was done in order to determine the feasibility of using convolutional neural networks in microstructure image processing, and to predict the desired property from image analyses.

The RVEs used by the CNN are \emph{raw} images of the material at particular length scales.
The researchers aimed to determine how the length scale of the RVE affected the accuracy of their model.
Note that the computer images are analyzed through discrete elements $s$ as image pixels.
The study used a completely different representation of RVEs from those used by finite element studies such as of \citeA{bone} and \citeA{cosserat}.
In the study it has been found that it is possible to perform image analysis on a small number of RVEs with a definite size (enough to distinguish between different $h$).
The researchers were able to only use seven images in the prediction of the property of interest with ``acceptable'' accuracy.

One of the problems with using advanced artificial intelligence methodologies, as admitted by \citeA{cnn}, was the uninterpretability of the model.
In order to circumvent this problem and to provide a basis for model classification and prediction, the researchers have provided ``saliency maps'' to see how the CNN classified the local states and inferred the interaction of such local states through implicit $\aleph$.
This may be a good protocol on studying how CNNs work internally and provide a one-way theories to predict desired properties.

Another problem posed by the use of CNNs, not mentioned in the research, is the unidirectional flow of information.
As evident from the research, the desired property, given different RVEs can be calculated; however, even with the introduction of saliency maps, it may still be difficult to reconstruct an RVE and its processing history from the properties desired.
This is the case with most engineering problems.

%%%%%%%%%%%%%%%%%%%%%%%%%%%%%%%%%%%%%%%%%%%%%%%%%%%%%%%%%%%%%%%%%%%%%%%%%%%%%%%%%%%%%%%%%%%%%%%%

\section{Materials knowledge systems}

\subsection{Microstructure functions}

The materials knowledge framework is a data science framework developed for the relatively simple modeling of material process, structure, and property.
This framework will be the modeling methodology of interest of the study.
A careful review of the data analysis framework will be given as follows.

The materials knowledge systems (MKS) framework rests upon the definition of microstructure functions.
This probability density function ($m(\vec{x}, h)$) has the following properties \cite{delin}:

\[
  m(\vec{x}, h)dH = \dfrac{^{h}V}{V}\biggr\rvert_{\vec{x}}
\]

This means that the probability density of finding a local state $dH = h$ can be expressed as the volume fraction of the $h$ within $V = dS$.
By summing over the local state space $H$, the probability should add up to $1$ by definition of the local state space.

\[
  \int_H m(\vec{x}, h)dH = 1
\]

From these descriptions, analogous expressions can be defined for discrete data.

\[
  \sum_{S} {^hm_s} = ^hVS \hspace{1in}
  \sum_{h=1}^{H} {^hm_s} = 1
\]

The probability function $^hm_s$ gives the volume fraction occupied (the expected probability) of the local state defined by $h$ in the voxel $s$.

With the definition of discrete microstructure functions, it is possible \emph{for computers} to perform calculations from various data relating to microstructure images.
Operations on probabilities and images may be applied.
Examples of these operations include calculation of spatial statisitics (\textit{N}-point spatial correlations), which have been used in quite a number of studies.
In each of these studies, there appears to be a different method for defining $h_i$, and hence $H$, depending on the properties being studied in materials.

Studies conducted by \citeA{yabansu14}, \citeA{delin}, \citeA{bunge}, and \citeA{sun17} have used different methods in defining the local state space depending on which definition will yield more \emph{computationally efficient} results.

\subsection{One-point spatial correlations}
\subsection{Two-point spatial correlations}
\subsubsection{Calculation of values}
\subsubsection{Low dimensional space representation}
\paragraph{Principal component analysis}
\paragraph{Partial least squares analysis}

%%%%%%%%%%%%%%%%%%%%%%%%%%%%%%%%%%%%%%%%%%%%%%%%%%%%%%%%%%%%%%%%%%%%%%%%%%%%%%%%%%%%%%%%%%%%%%%

\section{Julia}

\subsection{Comparison with currently used languages}
\subsection{Published computational studies and applications}
\subsubsection{NESSie.jl}
\subsubsection{JuMP.jl}
\subsubsection{BioSimulator.jl}
\subsubsection{QuantumOptics.jl}
