\chapter{Summary and Conclusions}

Zinc oxide thin films were fabricated using three different successive ionic layer adhesion and reaction processes.
The material structure was partially elucidated using characterization techniques such as optical microscopy, scanning electron microscopy, energy dispersive spectroscopy, and x-ray diffraction.
It was found out that the particle distribution among the three processes are random and do not exhibit significant differences.
From here, no stable relationships can be made between material phase distribution.
It was concluded that the difference in thin film property was due to the difference in crystal structure.
The project was unsuccessful in comparing the material state autocorrelations and material property.

\chapter{Recommendations}

It is recommended to study the effects of the crystal structures and orientations on the electrical properties of the fabricated material.
This means that there should be more than two material phases to be considered and studied using the MKS framework application.
This is because of the random and homogeneous distribution of crystalline material in the substrate.
Other thin film fabrication processes may be considered and studied.
