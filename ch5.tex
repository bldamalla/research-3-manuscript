\chapter{Summary and Conclusion}

Material structure of thin films using three different successive ionic layer adhesion and reaction processes.
It was found that the particle distributions among the three processes are random, and that they do not exhibit significant differences.
The four-point probe sensing data agrees with the randomness and sparseness of particle distribution.
Hence, there can be no strong dependence established between particle distribution and sheet resistance in the films fabricated processes.
Though there exists a correlation, it may be an overfit due to the nature of the data used.
It can be said that untreated glass impedes crystal growth when using ionic precursor solutions.

\chapter{Recommendations}

It is recommended to perform different fabrication methods by incorporating a drying process and treating glass substrates in boiling dilute sulfuric acid as discussed in Chapter 4 and to further develop the MKS framework in the Julia language.
The code used for the implementation is available on Github: bldamalla/research-3-adhesion.
